\documentclass{article}

% ==== RUSSIAN + XELATEX SETTINGS ====
\usepackage{polyglossia}
\setdefaultlanguage{russian}
\setotherlanguage{english}

\usepackage{fontspec}
% Красивые русские шрифты
\setmainfont{PT Serif}
\setsansfont{PT Sans}
\setmonofont{JetBrains Mono}

% ==== PACKAGES ====
\usepackage{amsmath, amssymb, amsfonts}
\usepackage{tikz}
\usetikzlibrary{automata,positioning}
\usepackage{algorithm}
\usepackage{algpseudocode}
\usepackage{fancyhdr}
\usepackage{extramarks}
\usepackage{geometry}

% ==== PAGE GEOMETRY ====
\geometry{
    a4paper,
    top=1.8cm,
    bottom=2cm,
    left=2cm,
    right=2cm
}

\linespread{1.12}
\setlength\parindent{0pt}

% ==== HOMEWORK INFORMATION ====
\newcommand{\hmwkTitle}{Лабораторная работа №1}
\newcommand{\hmwkDueDate}{Срок сдачи: 21 ноября 2025}
\newcommand{\hmwkClass}{Статистика, байесовский вывод, линейные и «деревянные» модели}
\newcommand{\hmwkClassTime}{}
\newcommand{\hmwkClassInstructor}{Преподаватели: Бойцев Антон, Волчек Дмитрий}
\newcommand{\hmwkAuthorName}{\textbf{Ластовецкий Дмитрий}}

% ==== HEADER / FOOTER (Только автор + название лабы) ====
\pagestyle{fancy}
\fancyhf{} % очистить все поля

\lhead{\hmwkAuthorName}
\rhead{\hmwkTitle}
\cfoot{\thepage}
\lfoot{\lastxmark}

\renewcommand\headrulewidth{0.4pt}
\renewcommand\footrulewidth{0.4pt}

% ==== HOMEWORK PROBLEM COUNTERS ====
\setcounter{secnumdepth}{0}
\newcounter{partCounter}
\newcounter{homeworkProblemCounter}
\setcounter{homeworkProblemCounter}{1}

\nobreak\extramarks{Задача \arabic{homeworkProblemCounter}}{}\nobreak{}

\newcommand{\enterProblemHeader}[1]{
    \nobreak\extramarks{}{Задача \arabic{#1} продолжается на следующей странице\ldots}\nobreak{}
    \nobreak\extramarks{Задача \arabic{#1} (продолжение)}{Задача \arabic{#1} продолжается на следующей странице\ldots}\nobreak{}
}

\newcommand{\exitProblemHeader}[1]{
    \nobreak\extramarks{Задача \arabic{#1} (продолжение)}{Задача \arabic{#1} продолжается\ldots}\nobreak{}
    \stepcounter{#1}
    \nobreak\extramarks{Задача \arabic{#1}}{}\nobreak{}
}

\newenvironment{homeworkProblem}[1][-1]{
    \ifnum#1>0
        \setcounter{homeworkProblemCounter}{#1}
    \fi
    \section{Задача \arabic{homeworkProblemCounter}}
    \setcounter{partCounter}{1}
    \enterProblemHeader{homeworkProblemCounter}
}{
    \exitProblemHeader{homeworkProblemCounter}
}

\renewcommand{\part}[1]{\textbf{\large Часть \Alph{partCounter}}%
\stepcounter{partCounter}\\}

% ==== MATH HELPERS ====
\newcommand{\solution}{\vspace{3mm}\textbf{\large Решение:}\\[2mm]}
\newcommand{\E}{\mathbb{E}}
\newcommand{\Var}{\mathrm{Var}}
\newcommand{\Cov}{\mathrm{Cov}}

% ==== DOCUMENT ====
\begin{document}

% ----- КРАСИВЫЙ ТИТУЛЬНИК -----
\begin{titlepage}
    \thispagestyle{empty}
    \vspace*{\fill}
    \begin{center}
        {\Large \hmwkClass}\\[0.6cm]
        {\huge \textbf{\hmwkTitle}}\\[0.8cm]
        {\large \hmwkDueDate}\\[0.4cm]
        {\large \hmwkClassInstructor}\\[1.5cm]
        {\large \hmwkAuthorName}
    \end{center}
    \vspace*{\fill}
\end{titlepage}

\setcounter{page}{2}

% ---------------------------------------------------------
% ---------------------- ЗАДАЧИ ---------------------------
% ---------------------------------------------------------

% === 1.1.1 ===
\begin{homeworkProblem}
Определить (с обоснованием), зависимы или независимы следующие события:
\begin{enumerate}
    \item[(a)] Несовместные события;
    \item[(b)] События, образующие $\sigma$-алгебру $\Sigma$ в пространстве $(\Omega,\Sigma,P)$;
    \item[(c)] События, имеющие одинаковую вероятность.
\end{enumerate}

\solution
\begin{enumerate}
    \item[(a)] Без ограничения общности возьмем два события и обозначим их за $A, B$. Так как события независимы, то по определению $A \cap B = \emptyset$, следовательно $P(A \cap B) = P(\emptyset) = 0$. По определению события независимы, если $P(A \cap B) = P(A) \cdot P(B)$. Далее, 
    \begin{itemize}
        \item если $P(A) = 0$ или $P(B) = 0$, то требуемое равенство выполняется и \textbf{события независимы}
        \item если же одновременно $P(A) > 0$ и $P(B) > 0$, то $P(A) \cdot P(B) > 0 = P(A \cap B)$ \textbf{и события не независимы}
    \end{itemize}
    Аналогично обобщается для большего количества событий. Если вероятность хотя бы одного из событий равна $0$, то требуемое равенство выполняется и события в совокупности независимы (что влечет попарную независимость). Если же вероятность всех событий положительна, то события в совокупности не независимы. 
    \item[(b)] Снова возьмем события $A, B$. Если на них образована сигма-алгебра, то независимо от взаимного положения $A, B$ (вложенности, отсутствия пересечений) все область $\Omega$ разбивается на $$E_1 = A \cap B, E_2 = A^c \cap B, E_3 = A \cap B^c, E_4 = A^c \cap B^c$$ и каждое из событий $E_1, E_2, E_3, E_4$ входит в сигма-алгебру как комбинация дополнений и объединений изначальных событий. Теперь заметим, что если $A, B$ независимы, то $$P(E_1) = P(A \cap B) \textbf{=}  P(A) \cdot P(B) $$ при этом $$P(A) = P(E_1 \cup E_3) = P(E_1) + P(E_3)$$ так как $E_1 \cap E_3 = \emptyset$. аналогично $$P(B) = P(E_1 \cup E_2) = P(E_1) + P(E_2)$$ и $$P(A) \cdot P(B) = (P(E_1) + P(E_3)) \cdot (P(E_1) + P(E_2)) = P^2(E_1) + P(E_1) P(E_2) + P(E_1) P(E_3) + P(E_2) P(E_3) $$ 
    для краткости обозначим $P(E_i) = p_i$. тогда события независимы если $$p_1 = p^2_1 + p_1p_2 + p_1p_3 + p_2p_3$$ $$p^2_1 + p_1p_2 + p_1p_3 + p_2p_3 - p_1 = 0 $$ $$p_1 ( p_1 + p_2 + p_3 - 1) + p_2p_3 = 0 $$ $$ p_1 p_4 = p_2p_3$$ $$P(A \cap B) \cdot P(A^c \cap B^c) = P(A^c \cap B) \cdot P(A \cap B^c)$$ То есть события $A, B$ \textbf{независимы если} $P(A \cap B) \cdot P(A^c \cap B^c) = P(A^c \cap B) \cdot P(A \cap B^c)$, \textbf{иначе не независимы}. 
    \item[(с)] Боюсь, в данном случае не получится сделать вывод о зависимости или независимости двух событий. Обозначим $A, B, P(A) = P(B) = p$. События независимы если $P(A \cap B) = p^2$. Мы можем подобрать соответствующие примеры: 
    \begin{itemize}
        \item отсутствия независимости двух событий. Возьмем один бросок стандартного шестигранного кубика и рассмотрим события  $$A = \{ 1, 3, 5\}, B = \{1, 4, 5\}$$ Тогда $$P(A) = P(B) = \frac{1}{2}; P(A\cap B) = \frac{1}{6}$$
        \item независимости двух событий. Возьмем два подряд бросания честной монетки и обозначим за $A$ выпадение орла в первом броске и за $B$  выпадение орла во втором броске. Тогда вероятности событий равны, и при этом $P(A \cap B) = P(A) \cdot P(B) = \frac{1}{4}$
    \end{itemize}
\end{enumerate}
\end{homeworkProblem}

% === 1.1.2 ===
\begin{homeworkProblem}
Опыт заключается в независимом подбрасывании двух симметричных
монет. Рассматриваются следующие события:
\begin{itemize}
    \item A -- появление герба на первой монете;
    \item B -- появление решки на первой монете;
    \item C -- появление герба на второй монете;
    \item D -- появление решки на второй монете;
    \item E -- появление хотя бы одного герба;
    \item F -- появление хотя бы одной решки;
    \item G -- появление одного герба и одной решки;
    \item H -- непоявление ни одного герба;
    \item K -- появление двух гербов.

\end{itemize}

Определить, каким событиям этого списка равносильны следующие события:

\begin{enumerate}
\item[(a)] $A + C$
\item[(b)] $AC$
\item[(c)] $EF$
\item[(d)] $G + E$
\item[(e)] $GE$
\item[(f)] $BD$
\item[(g)] $E + K$
\end{enumerate}

\solution
Рассмотрим пространство элементарных исходов
$$\Omega = \{\text{ОО}, \text{ОР}, \text{РО}, \text{РР}\}$$


Тогда события из условия можно записать так:
$$
\begin{aligned}
A &= \{\text{ОО}, \text{ОР}\}, &&\text{(орёл на первой монете)}\\
B &= \{\text{РО}, \text{РР}\}, &&\text{(решка на первой монете)}\\
C &= \{\text{ОО}, \text{РО}\}, &&\text{(орёл на второй монете)}\\
D &= \{\text{ОР}, \text{РР}\}, &&\text{(решка на второй монете)}\\
E &= \{\text{ОО}, \text{ОР}, \text{РО}\}, &&\text{(хотя бы один орёл)}\\
F &= \{\text{ОР}, \text{РО}, \text{РР}\}, &&\text{(хотя бы одна решка)}\\
G &= \{\text{ОР}, \text{РО}\}, &&\text{(один орёл и одна решка)}\\
H &= \{\text{РР}\}, &&\text{(нет орлов)}\\
K &= \{\text{ОО}\}. &&\text{(два орла)}
\end{aligned}
$$
откуда 

\begin{itemize}
    \item[(a)]
    \[
    A + C = A \cup C
    = \{\text{ОО}, \text{ОР}\} \cup \{\text{ОО}, \text{РО}\}
    = \{\text{ОО}, \text{ОР}, \text{РО}\} = E
    \]

    \item[(b)]
    \[
    AC = A \cap C
    = \{\text{ОО}, \text{ОР}\} \cap \{\text{ОО}, \text{РО}\}
    = \{\text{ОО}\} = K
    \]

    \item[(c)]
    \[
    EF = E \cap F
    = \{\text{ОО}, \text{ОР}, \text{РО}\} \cap \{\text{ОР}, \text{РО}, \text{РР}\}
    = \{\text{ОР}, \text{РО}\} = G
    \]

    \item[(d)]
    \[
    G + E = G \cup E = E
    \quad(\text{так как } G \subset E)
    \]

    \item[(e)]
    \[
    GE = G \cap E = G
    \]

    \item[(f)]
    \[
    BD = B \cap D
    = \{\text{РО}, \text{РР}\} \cap \{\text{ОР}, \text{РР}\}
    = \{\text{РР}\} = H
    \]

    \item[(g)]
    \[
    E + K = E \cup K = E
    \quad(\text{так как } K \subset E)
    \]
\end{itemize}

\end{homeworkProblem}

% === 1.1.3 ===
\begin{homeworkProblem}
Производится выстрел по вращающейся мишени, в которой закрашены два непересекающихся сектора по $20^\circ$ каждый. Найти вероятность попадания в закрашенную область.

\solution
Так как секторы не пересекаются, то закрашенный угол равен $40^\circ$. Площадь сектора линейно зависит от угла, следовательно, закрашена $\frac{1}{9}$ площади мишени. Так как вероятность попадания распределена равномерно по площади мишени, \textbf{она равна $\frac{1}{9}$}
\end{homeworkProblem}

% === 1.1.4 ===
\begin{homeworkProblem}
Два парохода должны подойти к одному и тому же причалу независимо друг от друга и равновозможно в течение суток. Определить вероятность того, что одному из них придется ожидать освобождения причала,
если время стоянки первого парохода -- 1 час, а второго -- 2 часа.

\solution
Обозначим за $t_1$ время прихода первого парохода к причалу, за $t_2$ время прихода второго парохода к причалу. Тогда кому-то придется ждать освобождения причала, если $[t_1, t_1 + 1] \cap [t_2, t_2 + 2]$. Рассмотрим полное дополнение этого события (что значит, что никакой пароход не ждал причал). Его можно задать как объединение событий $t_1 + 1 \leq t_2$ и $t_2 + 2 \leq t_1$. Приведем иллюстрацию: 
\begin{figure}[h]
    \centering
    \includegraphics[width=0.6\textwidth]{output.png}
    \caption{Области ожидания и отсутствия ожидания причала}
\end{figure}
Соответственно, никто никого не ждет с вероятностью, равной отношению соответствующей синей области к площади квадрата, то есть: $$P((t_1 + 1 \leq t_2) \cup (t_2 + 2 \leq t_1)) = \frac{\frac{23^2}{2} + \frac{22 ^2}{2}}{24^2} = \frac{1013}{1152}$$ Следовательно, искомая нами вероятность ожидания равна $$P([t_1, t_1 + 1] \cap [t_2, t_2 + 2]) = 1 - \frac{1013}{1152} = \frac{139}{1152} \approx 0.1206597$$

\end{homeworkProblem}

% === 1.1.5 ===
\begin{homeworkProblem}
Самолет, по которому ведется стрельба, состоит из трех различных по
уязвимости частей:
\begin{enumerate}
\item[(a)] Кабина летчика и двигатель
\item[(b)] Топливные баки
\item[(c)] Планер
\end{enumerate}
Для поражения самолета достаточно либо одного попадания в первую
часть, либо двух попаданий во вторую, либо трех в третью. При попадании в самолет одного снаряда, снаряд с вероятностью $p_1$ попадает в
первую часть, с вероятностью $p_2$ — во вторую, с вероятностью $p_3$ -- в
третью. Попавшие снаряды распределяются по частям независимо друг
от друга.
Известно, что в самолет попало $m$ снарядов. Найти условную вероятность $P(A|m)$ события $A$ -- "Самолет поражен" – при $m = 1, 2, 3, 4$

\solution
\begin{enumerate}
\item При $m = 1$ самолёт поражён тогда и только тогда, когда выстрел пришёлся в первую часть. Следовательно,
\[
P(A \mid m=1) = p_1.
\]

\item При $m = 2$ самолёт поражён тогда и только тогда, когда либо хотя бы один выстрел пришёлся в первую часть ($A_1$), либо оба выстрела попали во вторую часть ($A_2$).

Вероятность первого события найдём через дополнение:
\[
P(A_1) = 1 - P(A_1^c) = 1 - (1 - p_1)^2.
\]
Вероятность второго:
\[
P(A_2) = p_2^2.
\]
при этом
\[
A_1 \cap A_2 = \emptyset.
\]
Тогда
\[
P(A \mid m=2) = P(A_1) + P(A_2)
= 1 - (1 - p_1)^2 + p_2^2.
\]

\item При $m = 3$ удобнее сразу смотреть на дополнение. Самолёт \emph{не} поражён, если одновременно:
\begin{itemize}
  \item нет ни одного попадания в первую часть ($N_1 = 0$);
  \item во вторую попало строго меньше двух попаданий ($N_2 \le 1$);
  \item в третью попало строго меньше трёх попаданий ($N_3 \le 2$),
\end{itemize}
где $N_i$ — число попаданий в $i$-ю часть и $N_1 + N_2 + N_3 = 3$.

Из всех троек $(N_1,N_2,N_3)$ с суммой $3$ этим условиям удовлетворяет только
\[
(N_1,N_2,N_3) = (0,1,2),
\]
то есть самолёт не поражён только в случае когда один снаряд попал во вторую часть и два — в третью.

Тогда
\[
P(A^c \mid m=3)
= P(N_1=0, N_2=1, N_3=2)
= \frac{3!}{0!\,1!\,2!}\,p_1^0 p_2^1 p_3^2
= 3 p_2 p_3^2,
\]
следовательно
\[
P(A \mid m=3)
= 1 - P(A^c \mid m=3)
= 1 - 3 p_2 p_3^2.
\]

\item При $m = 4$ снова смотрим на дополнение. Чтобы самолёт не был поражён, нужно одновременно:
\[
N_1 = 0,\quad N_2 \le 1,\quad N_3 \le 2,\quad N_1 + N_2 + N_3 = 4.
\]
Но при $N_2 \le 1$ и $N_3 \le 2$ максимум возможная сумма $N_2 + N_3$ равна $1 + 2 = 3 < 4$, то есть таких конфигураций просто не существует. Значит
\[
P(A^c \mid m=4) = 0,\qquad
P(A \mid m=4) = 1.
\]
\end{enumerate}


\end{homeworkProblem}

% ========== 1.2 ==========

% === 1.2.1 ===
\begin{homeworkProblem}
Проверить, является ли
\[
f_\xi(x,y)=\frac{e^{-2|y|}}{\pi(1+x^2)}
\]
плотностью распределения случайного вектора.

\solution
Функция являетcz плотностью распределения случайного вектора, если она не отрицательна по $x, y$ и если ее интеграл на $\mathbb{R}^2$ равен 1. Первое очевидно по форме функции. Найдем интеграл $$
\iint\limits_{\mathbb{R}^2} f_{\xi}(x,y)\,dx\,dy
= \int_{-\infty}^{\infty} \int_{-\infty}^{\infty}
\frac{e^{-2|y|}}{\pi\bigl(1 + x^{2}\bigr)} \, dx \, dy.
= \int_{-\infty}^{\infty} \frac{1}{{\pi\bigl(1 + x^{2}\bigr)} } dx \int_{-\infty}^{\infty} e^{-2|y|} dy$$
Сразу заметим что левый интеграл это интеграл функции плотности стандартного распределения Коши, следовательно, он равен 1. Возьмем второй интеграл: $$\int_{-\infty}^{\infty} e^{-2|y|} dy = 2 \int_{0}^{\infty} e^{-2|y|} dy = 2 \int_{0}^{\infty} e^{-2y} dy = 2 \frac{e^{-2y}}{-2} \vert_0^\infty = -e^{-2y} \vert_0^\infty = 1$$

Значит, исходный интеграл равен 1. Следовательно, эта функция является плотностью распределения случайного вектора.

\end{homeworkProblem}

% === 1.2.2 ===
\begin{homeworkProblem}
Совместное распределение случайных величин $\xi$ и $\eta$ задано следующей таблицей:
\[
\begin{array}{c|ccc}
  \xi \backslash \eta & -1 & 0 & 1 \\ \hline
  -1 & \dfrac{1}{8} & \dfrac{1}{12} & \dfrac{7}{24} \\
  1  & \dfrac{1}{3} & \dfrac{1}{6}  & 0
\end{array}
\]

\begin{enumerate}
  \item[(a)] Найти маргинальные распределения $\xi$ и $\eta$.
  \item[(b)] Вычислить математическое ожидание, ковариационную и корреляционную матрицы вектора $(\xi,\eta)$.
  \item[(c)] Исследовать $\xi$ и $\eta$ на независимость и некоррелированность.
\end{enumerate}


\solution
\begin{enumerate}
    \item[(a)] для $\xi$:
        \[
        \begin{array}{c|c}
          \xi&   \\ \hline
          -1 & 0.5 \\
          1  & 0.5 
        \end{array}
        \]
        для $\eta$:
        \[
        \begin{array}{c|ccc}
          \eta & -1 & 0 & 1 \\ \hline
          & \dfrac{11}{24} & \dfrac{1}{4} & \dfrac{7}{24} 
        \end{array}
        \]
    \item[(b)] $\E (\xi,\eta) =  (\E\xi,\E\eta) = (0, -\dfrac{1}{6})$

Далее вычислим дисперсии. Сначала вторые моменты:
\[
\mathbb{E}\xi^{2}
= (-1)^{2}\cdot\frac12 + 1^{2}\cdot\frac12
= \frac12 + \frac12 = 1,
\]
\[
\mathbb{E}\eta^{2}
= (-1)^{2}\cdot\frac{11}{24} + 0^{2}\cdot\frac14 + 1^{2}\cdot\frac{7}{24}
= \frac{11}{24} + \frac{7}{24} = \frac{18}{24} = \frac34.
\]
Тогда
\[
\operatorname{Var}(\xi)
= \mathbb{E}\xi^{2} - (\mathbb{E}\xi)^{2}
= 1 - 0^{2} = 1,
\]
\[
\operatorname{Var}(\eta)
= \mathbb{E}\eta^{2} - (\mathbb{E}\eta)^{2}
= \frac34 - \left(-\frac{1}{6}\right)^{2}
= \frac34 - \frac{1}{36}
= \frac{27-1}{36}
= \frac{13}{18}.
\]

Теперь найдём смешанный момент $\mathbb{E}(\xi\eta)$:

\[
\mathbb{E}(\xi\eta)
= (-1)(-1)\cdot\frac{1}{8}
+ (-1)\cdot 0\cdot\frac{1}{12}
+ (-1)\cdot 1\cdot\frac{7}{24}
+ 1\cdot(-1)\cdot\frac{1}{3}
+ 1\cdot 0\cdot\frac{1}{6}
+ 1\cdot 1\cdot 0
\]
\[
= \frac{1}{8} - \frac{7}{24} - \frac{1}{3}
= \frac{3}{24} - \frac{7}{24} - \frac{8}{24}
= -\frac{12}{24} = -\frac{1}{2}
\]

Отсюда ковариация:
\[
\operatorname{Cov}(\xi,\eta)
= \mathbb{E}(\xi\eta) - \mathbb{E}\xi\,\mathbb{E}\eta
= -\frac{1}{2} - 0\cdot\left(-\frac{1}{6}\right)
= -\frac{1}{2}
\]

Ковариационная матрица вектора $(\xi,\eta)$:
\[
\mathbf{K} =
\begin{pmatrix}
\operatorname{Var}(\xi) & \operatorname{Cov}(\xi,\eta) \\[4pt]
\operatorname{Cov}(\xi,\eta) & \operatorname{Var}(\eta)
\end{pmatrix}
=
\begin{pmatrix}
1 & -\dfrac{1}{2} \\[4pt]
-\dfrac{1}{2} & \dfrac{13}{18}
\end{pmatrix}
\]

Коэффициент корреляции:
\[
\rho_{\xi,\eta}
= \frac{\operatorname{Cov}(\xi,\eta)}
{\sqrt{\operatorname{Var}(\xi)\operatorname{Var}(\eta)}}
= \frac{-\frac{1}{2}}{\sqrt{1\cdot \frac{13}{18}}}
= -\frac{1}{2}\sqrt{\frac{18}{13}}
= -\frac{3\sqrt{2}}{2\sqrt{13}}
\]
Тогда корреляционная матрица:
\[
\mathbf{R} =
\begin{pmatrix}
1 & \rho_{\xi,\eta} \\[4pt]
\rho_{\xi,\eta} & 1
\end{pmatrix}
=
\begin{pmatrix}
1 & -\dfrac{3\sqrt{2}}{2\sqrt{13}} \\[4pt]
-\dfrac{3\sqrt{2}}{2\sqrt{13}} & 1
\end{pmatrix}
\]

\item[(c)]
\textbf{Независимость.}
Если бы $\xi$ и $\eta$ были независимы, должно выполняться
\[
P(\xi=x,\eta=y) = P(\xi=x)\,P(\eta=y)
\quad\text{для всех }x,y.
\]
Проверим хотя бы одну пару, например $(\xi,\eta)=(-1,-1)$:
\[
P(\xi=-1,\eta=-1) = \frac{1}{8},
\]
\[
P(\xi=-1)P(\eta=-1)
= \frac{1}{2}\cdot\frac{11}{24}
= \frac{11}{48}.
\]
Так как
\[
\frac{1}{8} = \frac{6}{48} \ne \frac{11}{48},
\]
то условие независимости не выполняется. Следовательно, $\xi$ и $\eta$ \textbf{не независимы}.

\medskip
\textbf{Некоррелированность.}
Случайные величины некоррелированы, если
\[
\operatorname{Cov}(\xi,\eta)=0.
\]
но
\[
\operatorname{Cov}(\xi,\eta) = -\frac{1}{2} \ne 0,
\]
значит, $\xi$ и $\eta$ \textbf{коррелированы} 

\end{enumerate}
\end{homeworkProblem}

% === 1.2.3 ===

\begin{homeworkProblem}
Пусть имеются два одинаковых тетраэдра с числами $1,2,3,4$ на гранях. 
Подкидываем оба и смотрим на выпавшие числа $\xi_1$ и $\xi_2$. Зададим следующие
случайные величины:
\[
    \varphi_1 = \xi_1 + \xi_2, 
    \qquad
    \varphi_2 =
    \begin{cases}
        1, & (\xi_1\!:\!\xi_2)\cup(\xi_2\!:\!\xi_1),\\[2pt]
        0, & \text{иначе},
    \end{cases}
\]

\begin{enumerate}
  \item[(a)] Составить таблицу совместного распределения $\varphi_1$ и $\varphi_2$.
  \item[(b)] Найти маргинальные распределения $\varphi_1$ и $\varphi_2$.
  \item[(c)] Вычислить математическое ожидание, ковариационную и корреляционную матрицы вектора $(\varphi_1,\varphi_2)$.
  \item[(d)] Исследовать $\varphi_1$ и $\varphi_2$ на независимость и некоррелированность.
\end{enumerate}

\solution

Всего исходов $(\xi_1,\xi_2)$ $16$, все равновозможны c вероятностью $1/16$.

\begin{enumerate}
\item[(a)] Выпишем значения $(\varphi_1,\varphi_2)$:

\[
\begin{array}{c|cccc}
\xi_1\backslash\xi_2 & 1 & 2 & 3 & 4 \\ \hline
1 & (2,1) & (3,1) & (4,1) & (5,1)\\
2 & (3,1) & (4,1) & (5,0) & (6,1)\\
3 & (4,1) & (5,0) & (6,1) & (7,0)\\
4 & (5,1) & (6,1) & (7,0) & (8,1)
\end{array}
\]

По этим данным получаем совместное распределение $(\varphi_1,\varphi_2)$:

\[
\begin{array}{c|cc}
\varphi_1 \backslash \varphi_2 & 0 & 1 \\ \hline
2 & 0 & \dfrac{1}{16} \\
3 & 0 & \dfrac{1}{8} \\
4 & 0 & \dfrac{3}{16} \\
5 & \dfrac{1}{8} & \dfrac{1}{8} \\
6 & 0 & \dfrac{3}{16} \\
7 & \dfrac{1}{8} & 0 \\
8 & 0 & \dfrac{1}{16}
\end{array}
\]

\item[(b)] Маргинальные распределения:

\[
P(\varphi_1 = 2)=\frac{1}{16},\quad
P(\varphi_1 = 3)=\frac{1}{8},\quad
P(\varphi_1 = 4)=\frac{3}{16},\quad
P(\varphi_1 = 5)=\frac{1}{4},
\]
\[
P(\varphi_1 = 6)=\frac{3}{16},\quad
P(\varphi_1 = 7)=\frac{1}{8},\quad
P(\varphi_1 = 8)=\frac{1}{16}.
\]

Для $\varphi_2$:
\[
P(\varphi_2 = 1) = \frac{3}{4}, \qquad
P(\varphi_2 = 0) = \frac{1}{4}.
\]

\item[(c)] Вычислим математические ожидания и дисперсии.

\[
\E\varphi_1 = \sum_{k=2}^{8} k\,P(\varphi_1=k)
= 2\cdot\frac1{16}
+ 3\cdot\frac1{8}
+ 4\cdot\frac3{16}
+ 5\cdot\frac14
+ 6\cdot\frac3{16}
+ 7\cdot\frac1{8}
+ 8\cdot\frac1{16}
= 5.
\]

\[
\E\varphi_1^2 = \sum_{k=2}^{8} k^2\,P(\varphi_1=k)
= \frac{55}{2},
\qquad
\Var(\varphi_1) = \E\varphi_1^2 - (\E\varphi_1)^2
= \frac{55}{2} - 25 = \frac{5}{2}.
\]

Для $\varphi_2$:
\[
\E\varphi_2 = 0\cdot\frac14 + 1\cdot\frac34 = \frac34,
\quad
\E\varphi_2^2 = \frac34,
\quad
\Var(\varphi_2) = \E\varphi_2^2 - (\E\varphi_2)^2
= \frac34 - \left(\frac34\right)^2 = \frac{3}{16}.
\]

Теперь найдём $\E(\varphi_1\varphi_2)$:
\[
\E(\varphi_1\varphi_2)
= \sum_{k} \sum_{b} k b\, P(\varphi_1=k,\varphi_2=b)
= \sum_{k} k\cdot 1 \cdot P(\varphi_1=k,\varphi_2=1)
= 2\cdot\frac1{16}
+ 3\cdot\frac1{8}
+ 4\cdot\frac3{16}
+ 5\cdot\frac1{8}
+ 6\cdot\frac3{16}
+ 8\cdot\frac1{16}
= \frac{7}{2}.
\]

Ковариация:
\[
\Cov(\varphi_1,\varphi_2)
= \E(\varphi_1\varphi_2) - \E\varphi_1\,\E\varphi_2
= \frac{7}{2} - 5\cdot\frac34
= \frac{7}{2} - \frac{15}{4}
= -\frac{1}{4}.
\]

Ковариационная матрица вектора $(\varphi_1,\varphi_2)$:
\[
\mathbf{K} =
\begin{pmatrix}
\Var(\varphi_1) & \Cov(\varphi_1,\varphi_2)\\[4pt]
\Cov(\varphi_1,\varphi_2) & \Var(\varphi_2)
\end{pmatrix}
=
\begin{pmatrix}
\frac{5}{2} & -\frac{1}{4}\\[4pt]
-\frac{1}{4} & \frac{3}{16}
\end{pmatrix}.
\]

Коэффициент корреляции:
\[
\rho_{\varphi_1,\varphi_2}
= \frac{\Cov(\varphi_1,\varphi_2)}
{\sqrt{\Var(\varphi_1)\Var(\varphi_2)}}
= \frac{-\frac{1}{4}}{\sqrt{\frac{5}{2}\cdot\frac{3}{16}}}
= -\sqrt{\frac{2}{15}}.
\]

Тогда корреляционная матрица:
\[
\mathbf{R} =
\begin{pmatrix}
1 & -\sqrt{\dfrac{2}{15}}\\[4pt]
-\sqrt{\dfrac{2}{15}} & 1
\end{pmatrix}.
\]

\item[(d)] \textbf{Независимость.}
Если бы $\varphi_1$ и $\varphi_2$ были независимы, то
\[
P(\varphi_1=k,\varphi_2=b) = P(\varphi_1=k)\,P(\varphi_2=b)
\quad\text{для всех }k,b.
\]
Проверим, например, при $k=7$, $b=1$:
\[
P(\varphi_1=7,\varphi_2=1) = 0,
\]
в то время как
\[
P(\varphi_1=7)\,P(\varphi_2=1)
= \frac{1}{8}\cdot\frac{3}{4}
= \frac{3}{32} > 0.
\]
$\varphi_1$ и $\varphi_2$ \textbf{не независимы}

\medskip
\textbf{Некоррелированность.}
Поскольку
\[
\Cov(\varphi_1,\varphi_2) = -\frac{1}{4} \ne 0,
\]
случайные величины $\varphi_1$ и $\varphi_2$ \textbf{коррелированы}
\end{enumerate}

\end{homeworkProblem}


% === 1.2.5 ===
\begin{homeworkProblem}
Найти плотность распределения суммы двух независимых случайных величин
$\xi$ и $\eta$, если $\xi \sim \mathrm{Exp}_2$ и $\eta \sim U_{0,1}$.

\solution
Плотности исходных величин:
\[
f_\xi(x)=
\begin{cases}
2e^{-2x}, & x\ge 0,\\
0, & x<0,
\end{cases}
\qquad
f_\eta(y)=
\begin{cases}
1, & 0\le y\le 1,\\
0, & \text{иначе}.
\end{cases}
\]

Рассматриваем сумму $Z=\xi+\eta$. Так как величины независимы, используем свёртку:
\[
f_Z(z)=\int_{-\infty}^{\infty} f_\xi(z-y)f_\eta(y)\,dy.
\]

Одновременно должны выполняться условия $0\le y\le 1$ и $z-y\ge 0$, так что
интегрирование идёт по $y\in[0,\min\{1,z\}]$. Тогда рассмотрим по случаям : 
\begin{enumerate}
    \item[] {$z<0$} Область пуста, плотность равна нулю:
\[
f_Z(z)=0.
\]

    \item[] {$0\le z\le 1$} Тогда интегрируем от $0$ до $z$:
\[
f_Z(z)=\int_0^z 2e^{-2(z-y)}\,dy
=1-e^{-2z}.
\]

    \item[] {$z\ge 1$} Тогда ограничение даёт интеграл от $0$ до $1$:
\[
f_Z(z)=\int_0^1 2e^{-2(z-y)}\,dy
=(e^{2}-1)e^{-2z}.
\]
\end{enumerate}
    Итого 
\[
f_Z(z)=
\begin{cases}
0, & z<0,\\[4pt]
1-e^{-2z}, & 0\le z\le 1,\\[4pt]
(e^{2}-1)e^{-2z}, & z\ge 1.
\end{cases}
\]
\end{homeworkProblem}

\end{document}

